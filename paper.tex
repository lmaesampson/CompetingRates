\documentclass[nofootinbib,aps,pre,twocolumn,superscriptaddress,showkeys,showpacs]{revtex4-1}
\usepackage[capitalize]{cleveref}
\usepackage{amsmath, amsthm, amssymb}
\usepackage[toc,page]{appendix} 
\usepackage{graphicx}
\usepackage{subfigure}
\usepackage{wrapfig}
\usepackage{color}
\usepackage{lipsum}% http://ctan.org/pkg/lipsum
\usepackage{graphicx}% http://ctan.org/pkg/graphicx
\begin{document}

\title{Catchy Title Goes Here}
\author{Laura Sampson}
\affiliation{Center for Infectious Disease Dynamics, Pennsylvania State University, State College, PA, 16801}
\author{Matthew Ferrari}
\affiliation{Center for Infectious Disease Dynamics, Pennsylvania State University, State College, PA, 16801}

\begin{abstract}


\end{abstract}
\maketitle

\section{Introduction \label{sec:Intro}}
- Quantifying the size and age distribution of the population susceptible to measles is a critical tool in evaluating vaccination programs and developing interventions.  
- Unfortunately, direct observation of the susceptible population is challenging as it requires a sero-survey which can be cost-prohibitive.
- Previous efforts have estimated the susceptible distribution though demographic models that account for inputs through births and immunization via both vaccination an natural infection.
	- Winter et al (new paper on Madagascar)
	- Merler and the Italian group (http://www.thelancet.com/pdfs/journals/laninf/PIIS1473-3099(17)30421-8.pdf)
- The contribution of natural infection to population immunity can be challenging to estimate. In particular, the age-specific force of infection is not directly estimable from aggregate population time series ? (both papers referenced above rely on SIR models that assume a specific age-structured mixing) 

- The catalytic model has previously been used to estimate the age-specific force of infection from both serological surveys and the cross-sectional age-distribution of measles cases.  The catalytic model, fit to the cross-sectional age distribution of cases necessarily returns an estimate of the age distribution of susceptibles as a by-product.  This suggests that routine case surveillance could be used to generate an estimate of the age distribution of susceptibles.  

- Competing rates ? 
In the vaccine era, however, the age distribution of susceptibility is generated by the sum of the rates of vaccination and natural infection. 

Original applications of the catalytic model assumed that natural infection was the only source of immunity.  

Subsequent analyses have made simple assumptions about the role of vaccination, in particular that the delivery of vaccination is timely, and thus affects the age distribution of the susceptible population only by reducing the force of natural infection.  However, recent work has highlighted significant variability both in the maximum vaccination coverage achieved and the timeliness of vaccination ? that is, the proportion of children receiving vaccination after the recommended age.  Thus, failing to account for the age-specific pattern of vaccination may lead to a biased interpretation of case-data, as children who might be vaccinated later than the recommended age remain susceptible may contribute to incident cases. 

Further, vaccination does not necessarily imply immunization.  The efficacy of measles vaccine is often assumed to be ... (see Uzicanin article) though is also known to improve with age as older children are more likely to have lost maternally transferred antibodies (se many). The effectiveness of vaccine delivered in field settings may also vary dramatically due to stability and effectiveness of the vaccine cold chain.  A comparison of the age-distribution of vaccination and sero-prevalence may highlight areas with low effectiveness, and absent a sero-survey this assessment could be made using age-specific case records.  

Lay out the DRC case study?

Here we present a novel formulation of the catalytic model to estimate the age-specific sero-prevalence, and also the age-specific rates of vaccination and force of infection, and vaccine effectiveness using age-specific vaccine coverage data and case-records. We illustrate performance of this model using both simulated data, and measles case surveillance data from DRC combined with vaccination coverage surveys conducted as part of the 2013-14 DHS.  A contemporary measles sero-survey conducted during the 2013-14 DHS allows us to validate the performance of our estimates of sero-prevelence against direct measurements. We finally present a fit of the model to the surveillance data, vaccine coverage data, and sero-survey and discuss opportunities for combining data sources ? 


\section{Methods \label{sec:Methods}}
\subsection{Competing Rates Model \label{subsec:CompetingRates}}
The classic catalytic model for disease infection, developed in the late 1950's, gives the probability of immunity at age $a$ as
\begin{equation}
p(\mathrm{immune}|a) = 1 - \exp\left(-\int_0^a f(a') da'\right),
\label{eq:CatMod}
\end{equation}
where $f(a)$ is the \emph{force of infection} at age $a$, which can be thought of as the rate of infection at a particular age. This expression is valid in the absence of vaccination, as in this case infection is the only source of immunity. In the case of measles, this expression gives the probability of an individual at age $a$ being seropositive for measles.

In situations in which vaccination is present, vaccination provides a second means of acquiring immunity - the `force of vaccination,' or `vaccination hazard.' If we represent this as $v(a)$, then we can extend Eq.~\ref{eq:CatMod} to give the probability of an individual testing seropositive at age $a$ as
\begin{align}
p(\mathrm{immune}|a)  &= 1 - \exp\left( - \int_0^a f(a') + v(a') da'\right) \nonumber \\ 
&=1 - \exp\left(-\int_0^a f(a') - \int_0^a v(a') da'\right).
\label{eq:CatModSum}
\end{align}

The functional forms of $f(a)$ and $v(a)$ are free to be specified. In this study, we choose to use un-normalized Weibull distributions to parameterize both of these functions, as they have been shown to be sufficiently flexible to match a range of possible forces of infection and vaccination hazards. Thus $f(a) \rightarrow f(a|\mathbb{\psi})$ and $v(a) \rightarrow v(a|\mathbb{\theta})$, where $\mathbb{\psi}$ and $\mathbb{\theta}$ are vectors of the parameters we use for the Weibull distribution - height ($\eta$), scale ($\alpha$), and shape ($\beta$). This gives six parameters that fully specify the forms of $f(a)$ and $v(a)$ from Eq.~\ref{eq:CatModSum} - $\alpha$, $\beta$, and $\eta$ for two independent Weibull distributions. To allow for the fact that not all vaccinations produce immunity, we introduce a seventh parameter - the vaccine effectiveness ($\gamma$). This enters the equation as a multiplier on $v(a)$ that ranges between $0.0$ and $1.0$.

Besides the flexibility of the distribution, another appealing aspect of the Weibull as our choice of parameterization is that it can be integrated analytically, as
\begin{equation}
\int_0^a g(a';\eta,\alpha,\beta) = \frac{\beta}{\alpha} \eta \left( 1 - \exp(-(x/\alpha)^\beta) \right).
\end{equation}
We re-absorb the factor of $\beta/\alpha$ on the r.h.s. of this equation into the parameter $\eta$, and have a final expression for the probability of immunity at a particular age
\begin{align}
p(\mathrm{immune}|a) = 1 - \exp\big\{&-\eta_f\left(1-\exp(-(a/\alpha_f)^{\beta_f})\right)  \nonumber \\ 
&-\eta_v \gamma \left(1-\exp(-(a/\alpha_v)^{\beta_v}) \right)  \big\},
\end{align}
which is parameterized by seven total parameters - $\{\eta_f, \alpha_f, \beta_f, \gamma, \eta_v, \alpha_v, \beta_v\}$.

This expression gives the probability of observing a seropositive individual at age $a$, which accounts for one of our three datasets. The other two are vaccination data and case data. Using the function described above for vaccination hazard, the probability of an individual having been vaccinated by age $a$ is given by
\begin{equation}
p(\mathrm{vaccinated}|a) = 1 - \exp \left\{ - \eta_v (1-\exp\left(-(a/\alpha_v)^{\beta_v}\right) \right\}.
\end{equation}
Finally, the probability of an individual being recorded as a case at age $a$ is the force of infection at that age multiplied by the probability that an individual is susceptible at that age. The probability of susceptibility is of course $1 - p(\mathrm{immune}|a)$, and so the probability of observing a case at age $a$ is
\begin{widetext}
\begin{equation}
p(\mathrm{case}|a) = \left\{ 1 - \exp\left[ -\eta_f \left( \frac{\beta_f}{\alpha_f}\right)\left(\frac{a}{\alpha_f}\right)^{\beta_f - 1}\right] \exp\left[ (a/\alpha_f)^{\beta_f}\right] \right\}
\times \left\{ 1-p(\mathrm{immune}|a)\right\}.
\end{equation}
\end{widetext}

The data we work with (described in detail in Sec.~\ref{subsec:Data}) consists of cases as a function of age, serology tests and results as a function of age, and vaccination status as a function of age for a sample of individuals within a particular province. The likelihood for observing this dataset given values for the model parameters as
\begin{align}
\log \mathcal{L} (\mathbf{c}, \mathbf{v_t}, \mathbf{v_o}, \mathbf{s_t},&\mathbf{s_o}|\eta_f, \alpha_f, \beta_f, \gamma, \eta_v, \alpha_v, \beta_v)\nonumber \\ 
& = \sum_aB\left(v_o(a), v_t(a); p(\mathrm{vaccinated}|a)\right) \nonumber \\
&+ \sum_a B\left(s_o(a),s_t(a);p(\mathrm{immune}|a)\right) \nonumber \\
&+ M\left(\mathbf{c};p(\mathrm{case}|a)\right),
\label{eq:loglike}
\end{align}
where $s_t$ is the total number of individuals tested for IgM seropositivity, and $s_o$ is the number of positive tests; and $v_t$ is the number of individuals surveyed about vaccination status, while $v_o$ gives the number who have been vaccinated. $B$ represents a binomial probability, and $M$ represents a multinomial distribution, and the sums are over age classes. As noted, each data point includes the age of the individual in question.

\subsection{MCMC Sampler \label{subsec:MCMC}}
Given our set of model parameters and the definition of the likelihood in Eq.~\ref{eq:loglike}, we can generate samples of the posterior distributions of the model parameters using Markov chain Monte Carlo (MCMC) techniques. We use the \texttt{PTMCMC} sampler package in Python, which incorporates parallel tempering, differential evolution, and proposals along the eigenvectors of the covariance matrix. This sampler is described in detail in ~\cite{Arzoumanian2014}.

We must specify prior distributions on each of our model parameters before running. These are listed in Table~\ref{table:priors}. 
\begin{center}
\begin{tabular}{ c|c } 
Parameter & Prior \\
 \hline
 $\alpha_v$, $\alpha_f$ & cell2 \\ 
 $\beta_v$, $\beta_f$ & cell5 \\ 
 $\eta_v$, $\eta_f$& cell8 \\ 
 $\gamma$ & cell9 \\
 \hline
 \label{table:priors}
\end{tabular}
\end{center}

We run for $70000$ iterations, keeping every $20$th point in order to decrease autocorrelation. At the end of each run, we calculate the number of effective samples via thinning by the autocorrelation length, as
\begin{equation}
N_{eff} = \frac{N}{auto}.
\end{equation}
Both the effective number of samples and the autocorrelation lengths of all chains are shown in Table~\ref{table:autocorr}.

\subsection{Simulated Data\label{subsec:SimData}}
To confirm that we can accurately recover the force of infection and vaccination hazard using the model we have chosen, we generate simulated datasets consisting of age-specified measles case, vaccination, and serology data using a previously developed, age-structured MSIRV (Maternally immune, Susceptible, Infected, Recovered, Vaccinated) model~\cite{Metcalf2012}. We used demographic parameters from UN estimates for the DRC and increased vaccination linearly from 0 to 50\% over the first 30 years of a 50 year simulation. $R_{0}$ (the number of secondary cases resulting from the introduction of a single infected individual) was assumed to be constant at 15 over the entirety of the simulation. The force of infection (foi) was not directly chosen to be a Wiebull function, but is generated using the specified $R_0$ and WAIFW (Who Acquires Infection from Whom) matrix that describes social interactions as estimated via the POLYMOD study~\cite{Mossong2008}. This means that we cannot directly compare the recovered Weibull parameters to injected parameters, but we \emph{can} compare the recovered foi curve to the true values, as well as the recovered vaccination efficacy ($\gamma$) to the true value. 

While the raw MSIRV model output always entails a value for $\gamma$ of 1, we can simulate scenarios in which $\gamma$ is lower by drawing false positive vaccination responses from a binomial distribution with the corresponding probability, and adding these to the vaccination data generated from our simulation. For example, given that our simulation has a maximum vaccination rate of 50\%, we can simulate a $\gamma$ of 0.75 by assuming that, on average, 67\% of individuals in a given age class will report that they have been vaccinated.

After we generated a full time-series of case, vaccination, and serology data, we then downsampled the simulation results by randomly drawing the same number of observations as are present in the empirical data from the DRC.

\subsection{Data \label{subsec:Data}}

\subsection{Comparing Data and Inference \label{subsec:Comp}}

\section{Results \label{sec:Results}}

\bibliographystyle{unsrt}
\bibliography{../master}

\end{document}







